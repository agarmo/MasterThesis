%file included in thesis.tex


\chapter{Data Representation}
This section will describe the world representation used in the implementation.

\section{World Representation}
The world representation should have a number of qualities. First it must represent the
world in a satisfactory way. It should be easy to update and take up as little memory in
the robots memory as possible. Third it should be easy to understand for a human operator
without too much post-processing. 

When choosing how to represent the world one must take into account how much reasoning and
processing the robot should do with the sensor data. The more abstract world
representation, the more reasoning and processing must be done at the robot. On the
upside, abstract representations takes up less space in memory, and can be more
computationally effective when it comes to path finding and planning, than the less
abstract map representations. 

The problem when a robot has to reason, is that it might take the wrong decisions, and the
risk of erroneous map making are high. 

\subsection{Nodes}
The representation in this project is a kind of topological map. It is abstract in the way
that the world are classified into given nodes. Known features of the world are recognized
and represented as nodes in a graph. 

This means that the world need to be classified into predefined nodes before the mission
is started. In this project the nodes are junctions, or bends that are easily recognized
because of their fundamental difference to the straight pipe segments. 

The nodes should have a number of attributes describing how the node is different from the
other nodes. These attributes are:
\begin{itemize}
    \item Node type
    \item Number
    \item Time stamp when discovered
    \item Previous Node
    \item Distance since last node
    \item Orientation
    \item Number of edges and the angles of those
    \item List of anomalies
\end{itemize}

Most of this attributes are self-explaining, but the last attribute, \emph{list of
anomalies} deserves a closer explanation. 

Anomalies are those things that the robot should look after, and record. An anomaly are
everything that is not expected. In a pipe environment the surroundings are assumed to be
in some way, because they are designed that way. Everything that is not as expected should
be recorded for later inspections. 

This list will contain every anomaly from the previous node to the next one, where the
list is stored. The list will contain a time stamp of when it was discovered, distance
traveled from the previous node, and a remark why it was detected as an anomaly. 

Since the anomalies will be detected before the node which will contain the list are
detected, the list is temporary stored in the robots memory until the next node are
detected. 


\section{Map-building From the Sensor Data}


\subsection{Cylinder or Cone Fit}



\subsection{Edge-detection Algorithms}
Used in the stereo cam algorithm. Used for rectifying of the image from the web cams. 


\subsection{Other Feature Extraction}




