
\chapter{Introduction}
Making an autonomous robot have been the goal of scientists and engineers for years. This
is a difficult task which include many disciplines, and has up to date, not been solved in
a general way. Robots which can reason and take own decisions based on how it senses the
environment would greatly simplify our lives in many ways. Many tasks could be completely
outsourced to autonomous robots which does not tier, and is expendable. This means that it
can go places that a human could not. In short this can be summarized into the three Ds,
namely \emph{dull, dirty} and \emph{dangerous} work. Imagine an autonomous robot going
into a earthquake area and mapping out dangerous spots, and where injured people are,
before medical personnel enter the area. Or a robot mapping and inspecting the vast sewer
network of large city, searching for weak spots and clogged pipes which obstructs the
sewer flow. Or robots crawling the pipelines carrying water, oil or gas to detect leaks
which will cause catastrophic events if not tended to. This are all important work, which
either is dangerous, dirty or dull, or all three at once.

The problem when a robot is going into an unknown area and mapping the surroundings using
a pack of sensors is called \emph{SLAM}, short for Simultaneous Localization and Mapping.
This is a problem much researched around, and have been tried with numerous sensor types
and many algorithms are proposed. Some are best in structured environments, like office
landscapes and industrial areas, while other will preform better in more sparse areas with
less landmarks and recognizable features. 

In our global world the transport of resources, like oil and gas are of great importance. Much of this
transport is done using pipelines. There are millions of kilometers with pipelines around
the world. These pipelines need regular inspections to insure that the pipeline is intact
and undamaged. Also sewers, as mentioned above, needs inspections to keep a satisfactory
standard. Some of the pipes are considered accessible by humans, i.e. diameter over 80 cm,
but most are smaller than this, and uses various inspection devices feeding video to human
operator that must detect, record, and classify the damages. This is tedious work which
can be automated using robots. \cite{MAKRO-project}

One such robot suitable for pipe inspection might be the \emph{Pipe Inspection KOnda}
(PIKo) developed by SINTEF. This is a snake like robot, consisting a series of identical
modules interconnected by two degrees of freedom active joints. On each module there is a
set of four wheels, two near the ground and two on on the top of the modules. For
horizontal motion it uses these wheels and a train like motion. Since the joints are
connected with two degrees of freedom, the robot is capable of traversing vertical pipes
as well. This is achieved by spanning the pipe alternately and pushing the wheels towards
the pipe walls to gain enough friction for the robot to motion vertically. \cite{piko}

This report will concern the modelling and proposition of a navigation system which can be
used for pipe inspection. This navigation system must address a number of challenges. The


