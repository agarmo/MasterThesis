%file included in thesis.tex

\chapter{Implementation}
In this chapter implementation specific issues are treated. Figure
\ref{chap6:fig-implementation} shows the three-layer implementation of the project.
\begin{figure}[htbp]
    \centering
    \includegraphics[width=0.4\textwidth]{pics/implementation}
    \caption{Three-layer implementation}
    \label{chap6:fig-implementation}
\end{figure}
The \emph{low-level} layer reads and converts the raw sensor data into common data which can be
interpreted by the \emph{middle} layer. This layer handles all the ''washing`` of the
sensor data, and applies the reasoning and tries to recognize the environment. The third
and last layer is the \emph{world} layer which handles the global data, where the robot is
and where it should move next. 

The project is implemented in both \emph{C/C++} and Matlab. The sensor layer are
implemented in \emph{C/C++} while the other layers are implemented in Object-Oriented
Matlab. 

\section{Low-level Interfaces}
The low-level interfaces from the sensors are implemented in \emph{C/C++} and uses the
supplied sensor APIs. The SwissRanger 3000 API were used directly in Matlab. The Hokuyo
API was a little more tricky to use. A Matlab interface function were implemented to get
the sensor data directly into Matlab.

\subsection{Stereo Camera Implementation}
For the stereo camera, a program was written using the \emph{OpenCV} library. A open
source computer vision library, initially developed by Intel. This library have many
excellent and optimized functions for grabbing images, camera calibration, image rectification
and stereo matching. This produced a disparity map, which again where reprojected into 3D
by the \emph{cvReprojectTo3D()}-function. This coordinate images where saved to disk and
read into Matlab for further processing. 


\begin{algorithm}
\caption{The Stereo Matching Procedure}
    \begin{algorithmic}
    


    \end{algorithmic}
\end{algorithm}


\section{Mid-level Implantations}



\subsection{Analysis of 3D Sensor Data}



\subsubsection{Filtering of ToF Data}


\subsubsection{Surface Fit Algorithm}


\paragraph{Guass-Newton Least Squares}


\paragraph{Issues}
The algorithm is sensitive to edge data. This can be controlled with limiting the interval
in the z-direction of which the cylinder is fitted to the data.*************FIX*******


\section{High-level Implementations}

\subsection{World Representation}
The world representation are implemented as objects in Matlab.




