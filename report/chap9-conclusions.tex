
\chapter{Conclusions}
\label{chap9}
This report have proposed a navigation system for a pipe inspection robot using three
types of sensors, namely a Laser Range Finder, a Time-of-Flight camera and a stereo
camera. The different sensors have been investigated and their abilities have been
evaluated, and possible difficulties by using the proposed sensors have been evaluated.
This shows that the laser range finder provides reliable results in the plane without much
filtering and treatment of the sensor data. The time-of-flight camera on the other hand,
need more preparation before it can be utilized to create a map of the surroundings,
mostly because it provides more denser information, and this information is more prone to
noise. The stereo camera on the other hand did not preform that well in the given
environments, this is mostly due to the quality of the camera, and the lack of synthetic
lighting in the scene. Some enhancement of the images might have given better matching
results. But this were not tested in the report. 

A representation of the sensor data were also tested and implemented. This is a
topological representation of the world, which is based on much reasoning and
interpretation of the sensor data. A set of nodes are defined, this can be pipeline
junctions, like a pipeline bend, or a T-junction and likewise. It can also be any other
feature that is not like the usual featureless straight pipeline. As said, this relies
greatly on the interpretation of the sensors, which prove difficult with the proposed
algorithms. Although, no matching algorithms from the sensor data to the world
representation is proposed, the topological approach for the world representation were
chosen because of the simple and demanding less memory than other mapping approaches, like
the Occupancy Grid, which demand much memory, especially if implemented in three
dimensions. 


To draw conclusions from the results in Chapter \ref{chap7} and \ref{chap8} are difficult.
The proposed algorithms lacked selection and segmentation of the data. This was fatal
because the system needed to reason and the background of what it was looking for was
wrong. This shows the importance of selecting the right conditions and terms before
putting the data into the algorithms.

The representation although it was not tested properly in this thesis, have
the strengths that graph representations give. Path planning is
really just a shortest-path problem which can be implemented relatively easy in a fast and
efficient way. This also allows for easy interpretation back to the user, which then can
investigate the nodes which have been marked with anomalies. The backside of this is that
it requires much reasoning from the robot, which demand robust recognition algorithms.


This project can be seen as a mapping project on what to be on the look-out for when
designing this kind of system. This include, sensor calibration, sensor data
segmentation and recognition of environment.



\section{Further Work}


